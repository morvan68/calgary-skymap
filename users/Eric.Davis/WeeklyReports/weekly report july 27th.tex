\documentclass[11pt]{article}

\usepackage{graphicx, epsfig}
\usepackage{amsmath, amssymb, latexsym}
\usepackage[english]{babel}
\usepackage{amssymb}   
%\usepackage{graphicx}
%\usepackage{float} 






% Make a larger page (shrink the margins)
%
\setlength{\textwidth}{6.7in}
\setlength{\textheight}{9.0in}
\setlength{\evensidemargin}{0.0in}
\setlength{\oddsidemargin}{0.0in}
\topmargin -0.5in
\footskip 0.5in



\title{Weekly Report for July 27th}
\author{Eric Davis}
\begin{document}
\maketitle
\medskip



\section{Time Profiles}
\hspace{0.5cm}

I managed to improve the Gillam's missing data for a better Gaussian fit. Unfortunately, the Gaussfit function does not handle -NaN's well, and would not fit a curve with the missing data. However, I made missing data do an average of the data above and below it (in time), and it seems like a completely reasonable and accurate method. I updated all the Gillam time profiles. The comparison you can see in figure 1 and 2.

\begin{figure}[h!]
\includegraphics[scale=0.6]{gillam_jupiter_filled_data_11_20_2011.jpg}
\caption{An example of the Gillam data that was filled in with just zeroes. Messes up the Gaussfit enough to justify improving the code}
\end{figure}


\begin{figure}[h!]
\includegraphics[scale=0.6]{gillam_average_fill.jpg}
\caption{This is the fit with the data averaged. Seems more Gaussian and the program does not try and fit the zeros into the curvew}
\end{figure}


\section{Comparing the FWHM of the sites}

I compiled all the FWHM data from the four sites, as well as the heights of the profiles above the background, onto one spreadsheet. I then produced plots of FWHM's for all the filters except filter 6 (unreliable results) as a function of site latitude. Choosing site latitude was arbitrary, but it does demonstrate some disturbing results.  Pinawa consistently has a FWHM that is generally about 1 minute shorter than the other three sites, which are generally consistent with one another. The FWHM is always at least 30 seconds shorter than the other sites. We will need to discuss why this is the case, as currently I am fairly stumped what could be causing this besides an error with the optics of the imager. 

\begin{figure}[h!]
\includegraphics[scale=1.0]{filter0_FWHM.jpg}
\end{figure}

\begin{figure}[h!]
\includegraphics[scale=1.0]{filter1_FWHM.jpg}
\end{figure}

\begin{figure}[h!]
\includegraphics[scale=1.0]{filter2_FWHM.jpg}
\end{figure}

\begin{figure}[h!]
\includegraphics[scale=1.0]{filter3_FWHM.jpg}
\end{figure}

\begin{figure}[h!]
\includegraphics[scale=1.0]{filter4_FWHM.jpg}
\end{figure}

\begin{figure}[h!]
\includegraphics[scale=1.0]{filter5_FWHM.jpg}
\end{figure}

\begin{figure}[h!]
\includegraphics[scale=1.0]{filter7_FWHM.jpg}
\end{figure}


\section{Space Profiles}

I have also started a spreadsheet with space profiles of Jupiter. By having enough space profiles, in theory we should be able to calibrate with very little uncertainty the exact direction the all sky camera is pointed. However, Jupiter tends to occupy a very similar spot in the sky over short periods. I am trying to produce space profiles as far apart from each other as possible, but at this point in time it is somewhat limited by the data. Jupiter is generally only visible at night at the four sights from September to February, and data that's from over a year old often only has data from Gillam or Fort Smith. I am hoping that space profiles over approximately six months is enough to map out where the all sky camera is looking. 



\section{Future Considerations}
\hspace{0.5cm}

	There are several things I need to do in the coming weeks. First, I want to compare the sites filter by filter. However, I have a feeling that I will need to compile some weather data, as weather conditions can drastically change the light entering the camera. As well, I want to get space profiles for Jupiter over the course of a much larger timespan than the time profiles. It will take some work, but I should be able to calibrate what direction the camera is looking at if I can gather enough data. Next week, however, will mostly be focused on comparing the Gaussian features of Jupiter from the various sites.

\end{document}
\end{article}