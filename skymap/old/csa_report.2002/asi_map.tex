\documentclass[11pt,twoside]{article}   % pre-print version

\usepackage[active]{srcltx} % SRC Specials: DVI [Inverse] Search
\usepackage[dvips]{graphicx} % for including figures
\usepackage[dvips]{color}
\usepackage[square]{natbib}  % AGU style references

\usepackage{verbatim}   %to include other text files and source code listings
\usepackage{fancyhdr}   %to play with top of page headers
\usepackage{lastpage}   %to allow "page 1 of N"
\usepackage{longtable}  %for tables that span multiple pages
\usepackage{sectsty}    %to modify section headings
\usepackage{newcent}    %use this to get different looking output
%\usepackage{times}    %use this to get different looking output
\usepackage{fancybox}
\usepackage{multicol}
\usepackage{longtable}
\usepackage{colortbl}
\usepackage{lscape}  % to allow switching between portrait and landscape

\usepackage{makeidx}
\usepackage{showidx}
\usepackage{amsfonts}

\hfuzz5pt % Don't bother to report over-full boxes < 5pt
\vfuzz5pt % Don't bother to report over-full boxes < 5pt

% Make a larger page (shrink the margins)
%
 \setlength{\textwidth}{6.0in}
 \setlength{\textheight}{8.0in}
 \setlength{\oddsidemargin}{0.5in}
 \setlength{\evensidemargin}{0.5in}

 \setlength{\parindent}{0pt}  % don't indent paragraphs
 \setlength{\parskip}{1.3ex plus0.4ex minus0.2ex}  % space between paragraphs

% The  sectsty  package lets us play games with the section headings
 %\allsectionsfont{\sffamily\noindent\shadowbox}
 %\sectionfont{\sffamily\noindent\ovalbox}  % got carried away here
% \allsectionsfont{\color[gray]{0.9}\sffamily\noindent\shadowbox}
  \allsectionsfont{\color[gray]{0.1}\sffamily\noindent\colorbox[gray]{0.8}}

 \bibliographystyle{plainnat}
 \makeindex  % required to build index file

 \pagestyle{fancy}
 \lhead{\today}
 \chead{ASI Mapping}
 \rhead{\thepage\ of \pageref{LastPage}}
 \renewcommand{\headrulewidth}{0.4pt}

 \newcommand{\code}{\tt\footnotesize}

 \newcommand{\note}[1]{\marginpar{\small #1}}

% Magic to show gray background "DRAFT" on all pages
%
 \special{!userdict begin /bop-hook{gsave 200 30 translate 65
rotate /Times-Roman findfont 200 scalefont setfont 0 0 moveto 0.95
setgray (DRAFT) show grestore}def end}

% Use this to have section numbers in left margin.
%
 \makeatletter
 \def\@seccntformat#1{\protect\makebox[0pt][r]{\csname the#1\endcsname\quad}}
 \makeatother


\begin{document}
% -----------------------------------------------------------------
%  TITLE PAGE
% -----------------------------------------------------------------
 \begin{titlepage} \begin{center}
%
 \vspace*{0.1cm}
 \doublebox{ \begin{minipage}{6in} \centering

 \vspace*{1.2cm}
 \shadowbox{\parbox{4in}{\centering
 \vspace*{0.6cm}
 \parbox{3.5in}{\centering
 {\huge\sl NORSTAR \\[0.5cm] ASI Mapping \\[0.2cm] }}
 }}

 \vspace*{0.5cm}
  \includegraphics[width=4.2in]{../fig/composite_skymap.jpg.eps}
 \vspace*{0.5cm}

 \parbox{2in}{\centering \small
     Brian Jackel \\
     Institute for Space Research
     University of Calgary }
 %
  \vspace*{0.6cm}

 \parbox{3in}{\centering\small
  Canadian Space Agency \\
  Contract Report %\#XXX-YY-ZZZZ
  }
  \vspace*{0.6cm}

  \today
  \vspace*{0.5cm}
%
\end{minipage} }
%
 \end{center} \end{titlepage}

%
 \setcounter{tocdepth}{3}
 \tableofcontents \newpage
% \listoftables
% \listoffigures

%
%
\section*{Introduction}
%
An all-sky imager (ASI) measures the optical intensity of the sky
over a wide angular extent.  Modern systems use a charge-coupled
device (CCD) array detector to record the number of incident
photons during a given exposure interval.  This grid of
measurements is well suited for digital storage and certain kinds
of image processing. However, it is ultimately necessary to work
in more directly physical coordinates, such as geographic latitude
and longitude. Consequently, it is essential to ``map''
observations from pixel space to other reference frames and
coordinate systems.
 \\[2ex]
\noindent This report contains a presentation of the concepts and
formulae required to carry out several useful transformations.
Some practical methods for estimating orientation and instrumental
properties are also presented.
 \\[2ex]

 \begin{figure}[htb!]
 \begin{displaymath}
 \begin{array}{c}
  \fbox{CCD Pixels} \\
  {\Updownarrow} \\
  \fbox{Camera Optics} \\
  {\Updownarrow} \\
  \fbox{Latitude / Longitude} \Leftrightarrow \fbox{\parbox{3.9cm}{\centering Geographic \\ East / North / Zenith}} \Leftrightarrow \fbox{Azimuth / Elevation} \\
  {\Updownarrow}\\
   \fbox{\parbox{5.8cm}{\centering Celestial \\ Right Ascension / Declination}}
 \end{array}
 \end{displaymath}
  \caption{Important coordinate systems for all-sky imagers.
  \label{fig:flowchart}}
 \end{figure}

\noindent A summary of useful mappings is shown in figure
\ref{fig:flowchart}.  Although scientific application of ASI data
only requires conversion from pixel to geographic coordinates, a
further transformation to celestial coordinates is extremely
helpful for absolute determination of certain quantities. This is
due to the fact that all astronomical objects are essentially
fixed in the appropriate reference frame.  The ability to
transform ASI images to the celestial reference frame provides
confirmation that all quantities associated with camera optics and
orientation are accurately known. Conversely, unknown quantities
can be determined by requiring that a sequence of star frames
produce the appropriate results (i.e. stationary stars) when
transformed to celestial coordinates.

%

%%%%%%%%%%%%%%%%%%%%%%%%%%%%%%%%
 \newpage
 \section{Definitions}

\subsection{Matrix Formalism} \label{sec:matrix_formalism}

The problem of mapping ASI data is essentially two dimensional. A
CCD detects photon counts in a planar grid, while incident light
may be completely described in terms of zenith angle $\zeta$ and
azimuth $\alpha$ (or some other pair of angular coordinates).
Although the source of luminosity (i.e. the aurora) may have a
complex distribution as a function of distance from the imager,
none of this information is available from a single frame.

Despite the fundamentally 2-D nature of the problem, it is
advantageous to represent orientations in terms of 3-dimensional
vectors.  For example, the angular direction ($\zeta$,$\alpha$)
can be expressed\footnote{This convention is commonly used in
physics.  However, a different latitude/longitude system (defined
in Equations \ref{eqn:coordinates}) is also used in this report.}
in Cartesian coordinates
 \begin{eqnarray}
  x &=& r \sin\zeta \cos\alpha \\
  y &=& r \sin\zeta \sin\alpha \\
  z &=& r \cos\zeta
 \end{eqnarray}
which can be inverted to obtain the original angular coordinates
 \begin{eqnarray}
  \alpha &=& \tan^{-1}\left ( {y \over x } \right ) \\
  \zeta &=& \tan^{-1}\left ( {\sqrt{x^2 + y^2} \over z} \right )
 \end{eqnarray}
regardless of the value for $r$.  Using three Cartesian
coordinates to represent orientation requires 50\% more storage
space than a pair of angular coordinates, but has certain
computation advantages. For example, the angle between two vectors
is easily determined
 \begin{equation}
  \cos\gamma = {{\mathbf{a} \cdot \mathbf{b}}\over{|\mathbf{a}| |\mathbf{b}|}}
 \end{equation}
More importantly, many useful transformations can be expressed in
terms of multiplication with a 3-by-3 matrix
 \begin{eqnarray}
  \mathbf{a}_2 &=& \mathbf{T} \, \mathbf{a}_1 \\
 \left [ \begin{array}{c} x \\ y \\ z \end{array} \right ]_2 &=&
  \left [\begin{array}{ccc}
  T_{11} & T_{21} & T_{31} \\
  T_{12} & T_{22} & T_{32} \\
  T_{13} & T_{23} & T_{33}
 \end{array} \right ]
  \left [ \begin{array}{c}  x \\ y \\ z \end{array} \right ]_1
 \end{eqnarray}
Inverse transforms of this class are easily accomplished by
multiplication with the matrix inverse
 \begin{equation}
   \mathbf{a}_1 = \mathbf{T}^{-1} \, \mathbf{a}_1
 \end{equation}
where for orthonormal matrices (such as those corresponding to
rotations)
 \begin{equation}
  \mathbf{T}^{-1} = \mathbf{T}^T
 \end{equation}
the inverse is equal to the matrix transpose.  Multiple rotations
can be accomplished by successive multiplication with the
appropriate matrices.  When doing this it is important to remember
that matrix multiplication is associative
 \begin{equation}
  \left ( \mathrm{T}_3  \mathrm{T}_2 \right ) \mathrm{T}_1 =  \mathrm{T}_3 \left ( \mathrm{T}_2
   \mathrm{T}_1  \right )
 \end{equation}
but not multiplicative
 \begin{equation}
  \mathrm{T}_2 \mathrm{T}_1 \ne \mathrm{T}_1 \mathrm{T}_2
 \end{equation}

\subsection{Celestial Coordinates}

As noted in the introduction, astronomy is not directly relevant
to the problem of relating ASI frames to geographic coordinates.
There are, however, benefits to be gained from exploring this
topic.  Stars in the ASI frames provide an indication that viewing
conditions are clear.  They can also be used as reference sources
to determine camera response and orientation, which are the
central topics of this report.

Precisely determining the orientation of astronomical objects
relative to a terrestrial observer is not trivial.  Fortunately,
the requirements for ASI applications are much less demanding than
for astronomy.  Mapping the entire $180^\circ$ angle of sky to a
$512\times 512$ element CCD produces a fundamental lower
resolution limit of approximately 20 arc minutes.  In practice,
multi-pixel binning may reduce this by another factor of two.
Camera optics will also have a non ideal point-spread-function,
further limiting the achievable angular resolution.  Consequently,
the angular accuracy required for this report is no more than
approximately 6 arc minutes ($0.1^\circ$).  This will be more than
adequate for ASI data while allowing for a considerable
simplification in algorithmic complexity.

Complete formula for conversion between celestial and terrestrial
coordinates can be found in standard references (for example,
\cite{lang_1980}). A simplified version of the necessary equations
are presented by \cite{hapgood_1992} in the context of general
space physics coordinate conversions. In this notation, conversion
from celestial to earth centered coordinates (GEI to GEO) can be
expressed as a rotation about the z-axis common to both systems.
This rotation is in the plane of the Earth's geographic equator
from the first point of Aries\footnote{From \cite{lang_1980}:
``{\it Vernal equinox}: that point of intersection between the
ecliptic and the celestial equator which occurs when the Sun is
going from south to north. The equinox has the symbol $\gamma$ and
is sometimes called the first point of Aries.''}
%
to the Greenwich Meridian. The rotation angle $\theta$ is the
Greenwich Mean Sidereal Time given by
 \begin{equation}
  \theta = 100.461 + 36000.770 T_0 + 15.04107 UT
 \end{equation}
where $T_0$ is the time in Julian centuries (36525 days) from
12:00 UT on 1 January 2000 (known as epoch 2000.0) to the previous
midnight and $UT$ is the universal time.  Conversion of a GEI
Cartesian vector into the GEI system is accomplished by
multiplication with the rotation matrix
 \begin{equation}
 {\mathbf T_1} = \left[
 \begin{array}{ccc}
  \cos\theta & \sin\theta & 0 \\
  -\sin\theta & \cos\theta & 0 \\
  0 & 0 & 1
 \end{array}
 \right]
 \end{equation}
and the reverse transformation by multiplication with $T_1^{-1} =
T_1^T$.



 \subsubsection{Parallax}
 \label{sec:parallax}

As given in the 1996 Astronomical Almanac, B61, the corrections
for diurnal parallax (due to a displacement for the observer from
the center of the earth) of distant objects are
 \begin{eqnarray}
  \Delta\alpha &=& \pi (\rho \cos\varphi \sin H \sec\delta) \\
  \Delta\delta &=& \pi (\rho\sin\varphi\cos\delta - \rho\cos\varphi\cos H \sin\delta)
 \end{eqnarray}
where $\rho$ is the geocentric distance in units of the Earth's
equatorial radius, $\varphi$ is the geocentric latitude, $H=
\theta_0 - \alpha$ is the local hour angle, and $\pi$ may be
calculated from $8''.794$ divided by the geocentric distance of
the body in astronomical units.  These will be very small
corrections and can be neglected in most cases.  For the moon (and
other very close bodies such as satellites) more precise formulae
are required.



\subsection{Geodetic and Geocentric Coordinates}
%
Relating ASI measurements to astronomical sources requires only
the use of geocentric (earth-centered spherical) coordinates.
However, an understanding of geodetic coordinates is required to
calculate geocentric site locations from commonly available
information and to convert to geographic latitude and longitude.

For the purposes of this report, the term ``geocentric'' will be
used to refer to a Cartesian reference frame with the z-axis
aligned with the geographic north pole (Earth's rotation axis) and
the x-axis defined by the intersection of the Greenwich Meridian
and the geographic equator.  The y-axis is then defined by the
requirement that the coordinate system be right-handed.  It may
also be useful to work with spherical coordinates, which will be
explicitly defined below.

The term ``geodetic'' will be used in this report to denote a
system which partially accounts for the (slightly) non-spherical
shape of the earth. From \S 24.1 of
\cite{geospace_handbook_geodesy}:
\begin{quotation}
The geoid is the equipotential surface in the earth's gravity
field that coincides most closely with the undisturbed mean sea
level extended continuously under the continents.  The direction
of gravity is perpendicular to the geoid at every point.  On the
earth's surface, this direction is defined by two angles, the
astronomical coordinates: The astronomical latitude is the angle
$\Phi$, $-90^\circ \le \Phi \le 90^\circ$, that the gravity vector
forms with the equatorial plane; and the astronomical longitude is
the angle $\Lambda$, $0^\circ \le \Lambda \le 360^\circ$ and
positive eastwards, that the plane defined by this vector and the
celestial pole forms with Greenwich meridional plane.

The reference ellipsoid is a a simple mathematical figure that
closely approximates the geoid, historically on a regional basis,
but on a global basis for modern requirements.  It is a surface of
revolution formed by rotating an ellipse about its minor
(vertical) axis resulting in an oblate (flattened at the poles)
ellipsoid.
\end{quotation}

Definition of geodetic coordinates requires the adoption of some
reference ellipsoid.  Two examples are given in table
\ref{tab:referencegeoids}; the most useful of these is WGS84, used
for GPS satellite positioning systems.  It differs only slightly
from the International Astronomical Union (IAU) ellipsoids.

 \begin{table}[hbt!]
 \hrule
 \centering3
 \begin{tabular}{lclllr}
   {} & {}  & IAU 1976 & WGS 84 (GPS) & \\
   semi-major axis & $a$  & $6378.140$ km & $6378.137$ km & equatorial radius\\
   flattening & $f$  & $1/298.257$ & $1/298.257223563$ \\
   semi-minor axis & $b$  & $6356.755$ km & $6356.752$ km &$b=a(1-f)$\\
   eccentricity & $e$  & $8.181921 \times 10^{-2}$ & $8.181919 \times 10^{-2}$ &$1-e^2 = (1-f)^2$\\
 \end{tabular}
 \hrule
  \caption{Parameters for standard reference geoids.
  \label{tab:referencegeoids}}
  \hrule
\end{table}

In most cases, values of latitude and longitude are implicitly
given in geodetic coordinates, and conversion to geocentric is
required before further transformations. The following
equations\footnote{Note the difference between the convention used
here and in \S \ref{sec:matrix_formalism}.} relate geocentric
(Cartesian and spherical) and geodetic coordinates:
%
 \begin{eqnarray}
 \mathbf{Geocentric} & \mathbf{Geocentric} & \quad\mathbf{Geodetic} \nonumber \\[-5pt]
 \mathbf{Cartesian} & \mathbf{Spherical} & \quad\mathbf{Ellipsoidal} \nonumber \\
  X_{GEO} =& R \cos\varphi \cos\lambda &= (N+h) \cos\phi \cos\lambda   \nonumber\\
  Y_{GEO} =& R \cos\varphi \sin\lambda &= (N+h) \cos\phi \sin\lambda  \qquad \\
  Z_{GEO} =& R \sin\varphi &=  \left( {b^2 \over a^2} N +h \right) \sin\phi   \nonumber
 \label{eqn:coordinates}
 \end{eqnarray}
%
Longitude $\lambda$ is the same in both systems, $\varphi$ is the
\emph{geocentric} latitude for a spherical earth and $\phi$ is the
\emph{geodetic} latitude (geographical latitude on the ellipsoid).
R is the spherical earth radius (6371.2 km), $h$ is the local
height above the reference ellipsoid, and $N$ is the east-west
radius of curvature (which is a function of latitude)
%
 \begin{equation}
  N(\phi) = {{a^2}\over \sqrt{a^2 \cos^2\phi + b^2 \sin^2 \phi}}
 \end{equation}

The inverse transform is more complicated. The quantity
 \begin{equation}
  p = \sqrt{X_{GEO}^2 + Y_{GEO}^2} = (N+h) \cos\phi
 \end{equation}
can be directly calculated from Cartesian coordinates, and also
provides a useful expression for the geodetic height
 \begin{equation}
   h(\phi) = {p \over \cos\phi} - N(\phi)
 \end{equation}
Geodetic latitude can be determined by successive approximation of
 \begin{equation}
  \tan\phi_{(i+1)} = {Z_{GEO}\over p} \left(1-e^2 {N(\phi_{(i)}) \over {N(\phi_{(i)}) + h(\phi_{(i)}) }} \right)^{-1} \\
 \end{equation}
with a starting value $\phi_{0} = \tan^{-1} \left( Z_{GEO} / p
\right) = \varphi $.  In practice this converges after a few
iterations.

 \subsubsection{Local Orientation}

The geodetic eastward direction expressed in Cartesian
coordinates is easily calculated
 \begin{equation}
   \widehat{\mathbf{east}} = {\partial \over \partial \lambda}
   [X,Y,Z]_{GEOC} = [ -\sin\lambda, \cos\lambda, 0]
 \end{equation}
as is the zenith (vertical)
 \begin{equation}
   \widehat{\mathbf{zenith}} = {\partial \over \partial h}
   [X,Y,Z]_{GEOC} = [ \cos\phi\cos\lambda, \cos\phi\sin\lambda, \sin\phi]
 \end{equation}
Direct calculation of the northward vector is less trivial, but
can be easily obtained from a cross product
 \begin{equation}
  \hat{y} = \widehat{\mathbf{north}} = \widehat{\mathbf{zenith}}
  \times \widehat{\mathbf{east}} = [ -\sin\phi \cos\lambda, -\sin\phi \sin\lambda, \cos\phi]
 \end{equation}

Explicitly, the relationship between local geodetic and cartesian
geocentric coordinates is given by:
%
 \begin{equation}
 \left [ \begin{array}{c}
  x \\ y \\ z
 \end{array} \right ]_{geoc}
 =
 \left [ \begin{array}{ccc}
  -\sin\lambda & -\sin\phi\cos\lambda & \cos\phi\cos\lambda \\
  \cos\lambda  & -\sin\phi\sin\lambda & \cos\phi\sin\lambda \\
  0            &  \cos\phi            & \sin\phi
 \end{array} \right ]
  \left [ \begin{array}{c}
  east \\ north \\ zenith
 \end{array} \right ]_{geod}
 \end{equation}
%
where as usual the inverse transform is provided by the matrix
transpose.

 \subsubsection{The Celestial Pole}

By definition, the GEI vector from the center to the earth towards
the celestial north pole is $[0,0,\infty]$.  For observations from
other locations, the effect of an offset vector ( may have to be
included
 \begin{equation}
  \left [ \begin{array}{c}
   x \\ y \\ z
   \end{array} \right ]_{local}
 =
  \left [ \begin{array}{c}
   x \\ y \\ z
   \end{array} \right ]_{GEI}
   +
   \left [ \begin{array}{c}
   x \\ y \\ z
   \end{array} \right ]_{position}
 \end{equation}
with local azimuth and elevation determined for the offset vector.
This degree of detail is only necessary for celestial objects that
are relatively close, as noted in \ref{sec:parallax}.
Consequently, the unit vector pointing towards the celestial pole
is very well approximated by $[0,0,1]_{gei}$ for any location on
the earth.  At all times this transforms to $[0,0,1]_{geoc}$ which
in turn becomes
 \begin{equation}
  \left [ \begin{array}{c}
   0 \\ 0 \\ 1
   \end{array} \right ]_{geoc}
 =
  \left [ \begin{array}{c}
   0 \\ \cos\phi \\ \sin\phi
   \end{array} \right ]_{geod}
 =
  \left [ \begin{array}{c}
   east \\ north \\ zenith
   \end{array} \right ]_{geod}
 \end{equation}
so the local azimuth is always zero, and the zenith angle equal to
the co-latitude $\bar\phi = 90^\circ - \phi$.

 \subsection{Instrument Orientation}
  \label{ssec:cameraorientation}

During field installation it is desirable to orient an ASI so that
it is pointing vertically and aligned with north at the top of the
image. Vertical levelling to a few degrees is easily accomplished.
Azimuthal alignment is more difficult, unless orientation of the
building or mount system is precisely known. Consequently, some
general method of accounting for non-ideal alignment is required.

Starting with a vector in some Cartesian coordinate system
$\{x_1,y_1,z_1\}$, it is often useful to carry out a rotation
about one axis by some angle $\psi$ to produce the new vector
$\{x_2,y_2,z_2\}$. This can be expressed as matrix multiplication
by some orthonormal array $\mathbf{R}$
%
 \begin{equation}
  X_2 = \mathbf{R}(\psi,\mathrm{axis}) \: X_1
 \end{equation}

Multiple rotations can be accomplished by successive
multiplication by the appropriate matrices. The transformation
from any given Cartesian coordinate system to another can be
achieved by three successive rotations performed in a specific
sequence. One common convention uses three \emph{Euler angles}
applied to the $z_1$, $x_2$, and $z_3$ axes.  However, as noted by
\cite[pp 147-148]{goldstein_1980} these systems
\begin{quote}
 \ldots have the drawback that when the primed coordinate system
 is only slightly different from the unprimed system, [certain
 angles] become indistinguishable, as their respective axes of
 rotation{\ldots}are then nearly coincident.  To get around this
 problem all three rotations are taken around different axes.  The
 first rotation is about the vertical axis and gives the {\it
 heading} or {\it yaw} angle $\phi$.  The second is about a perpendicular
 axis fixed in the vehicle and normal to the figure axis; it is
 measured by the {\it pitch} or {\it attitude} angle $\theta$.  Finally
 the third angle is one of rotation about the figure axis of the
 vehicle and is the {\it roll} or {\it bank} angle $\psi$.
\end{quote}
%
The resulting rotation matrix\footnote{Care must be taken to avoid
typographical errors when coding the full $\mathbf{T}_3$ matrix.
For languages where matrix multiplication is built-in (such as
{\sf IDL}), it may be simpler to construct and multiply the three
elementary rotation matrices.} %\small
%
 \begin{eqnarray}
%
 \mathbf{T}_3 &=& \mathbf{T}(\psi,x_2) \mathbf{T}(\theta,y_1)
 \mathbf{T}(\phi,z_0) \\
  &=& \left[
 \begin{array}{ccc}
  \cos\theta\cos\phi &
  \cos\theta\sin\phi &
  -\sin\theta \\
 %
  \sin\psi\sin\theta\cos\phi - \cos\psi\sin\phi &
  \sin\psi\sin\theta\sin\phi + \cos\psi\cos\phi &
  \cos\theta\sin\psi \\
 %
  \cos\psi\sin\theta\cos\phi + \sin\psi\sin\phi &
  \cos\psi\sin\theta\sin\phi - \sin\psi\cos\phi &
  \cos\theta\cos\psi \\
  %
 \end{array}
 \right] \nonumber
%
 \end{eqnarray} \normalsize
can be used to transform between two arbitrary coordinate systems
$X$ and $X^\prime$. This is a natural system for ground based
work.

An ideally mounted camera would be aligned with the local north,
east, and vertical vectors.  In practice it is usually possible to
orient the camera within a few degrees of vertical, but exact
north/south alignment is more difficult, and may be impossible due
to physical constraints (i.e. building layout).  As well, the CCD
alignment may not be consistent with the external camera mounting.
Typical values for ``yaw'' may therefore be large, while ``pitch''
and ``roll'' are expected to be small. Hence, the transformation
from local geodetic orientation begins with a rotation about the
zenith ($z_1$).  This could involve angles as large as $\pm
180^\circ$. Subsequent rotations about the $y_2$ and $x_3$ axes
are expected to be small if the instrument is nearly level.

%Some kind of camera system will be placed in front of the
%detector, and the entire apparatus will be attached to some
%mounting system.  It is crucial to determine the relationship
%between camera optical angles and locations on the detector, as
%well as orientation of the camera with respect to the real world.
%
%Section \ref{ssec:cameraoptics} will explore the mapping between
%optics and the CCD.   For this we will assume that the CCD is
%aligned perfectly flat in the camera focal plane.  Section
%\ref{cameraorientation} will present issues associated with camera
%orientation.

%%%%%%%%%%%%%%%%%%%%%%%%%%%%%%%%%%%%%%%%%%%%%%%%%%%%%%%%%%%%%%%%%%%%%
%
 \subsection{Optical System}

Many optical systems, including the NORSTAR ASIs, are circularly
symmetric about an ``optical axis''.  Incoming light rays arrive
at some angle $\gamma$ with respect to the optical axis, and are
fully specified in terms of a second polar angle $\psi$.  After
passing through the camera optics the ray will exit in some new
direction ($\gamma^\prime, \psi^\prime$).


 \subsubsection{Image Rotation}
For the purposes of this report it will be
assumed\footnote{Instrument schematics indicate that images should
not be reversed, but including a reversal produces produces small
values of yaw for well aligned instruments.} that the polar angle
is reversed ($\psi= - \psi^\prime$) by the camera optics. This is
true for many imaging systems in which the output plane is
mirrored about the $x$ and $y$ axes. The matrix representation of
this effect
 \begin{equation}
  \mathbf{T} = \left [ \begin{array}{ccc}
  -1 & 0 & 0 \\
  0 & -1 & 0 \\
  0 & 0 & 1
  \end{array} \right ]
 \end{equation}
is equivalent to a $180^\circ$ rotation about the optical ($z$)
axis.  This could be folded into the instrument frame rotation
from the previous section, but for this report will be explicitly
included in the optical transformations.


 \subsubsection{Image Scaling}
Changing $\gamma$ to $\gamma^\prime$ has the effect of expanding
or contracting the solid angle subtended by the image.  For an ASI
the goal is a reduction of the $2\pi$ steradian all-sky field of
view onto an angular extent approximately equal to
 \begin{equation}
  \Omega \approx {{L_x \times L_y} \over f}
 \end{equation}
where $f$ is the focal length from the optics to the CCD, and
$L_x,L_y$ are the CCD dimensions.

A lens that covers a hemispherical field of view (180 degrees) is
usually called a fish-eye lens.  These lenses have inherent large
distortion because it is not possible to form an image of a
hemispheric field onto a plane without distortion.  This
distortion should not be considered as an aberration, but as a
necessary result of the projection\footnote{Much of the
information in this section was taken from notes provided by
Robert Eather of Keo Consultants.}.

There are five standard projection systems that have been used in
fish-eye lens design.  One convention for representing these
expresses the linear displacement in the image plane $r$ in terms
of the system focal length $f$ and initial off-axis angle
$\gamma$. However, the relationship between $r$ and $\gamma$ may
be highly nonlinear, and not amenable to simple modelling.
Consequently, the convention used here expresses the final
off-axis angle in terms of linear displacement and focal length
 \begin{equation}
  {r\over f} = \tan\gamma^\prime
 \end{equation}
The equations governing the five standard projections are then
%
 \begin{eqnarray}
  \tan\gamma^\prime &=& \tan\gamma  \mbox{\hspace{5.9ex}normal camera lens}\\
   &=& 2 \tan{\gamma\over 2}  \mbox{\hspace{3.9ex}stereographic} \\
   &=& \gamma  \mbox{\hspace{9.35ex}equidistant}\\
   &=& \sin\gamma  \mbox{\hspace{6.35ex}orthographic} \\
   &=& 2\sin{\gamma\over 2} \mbox{\hspace{4.45ex}equisolid}
 \end{eqnarray}
%
The left panel of Figure \ref{fig:fisheyefunctions} shows all five
of these functions in terms of entrance angle and normalized
linear displacement.

Almost all commercially available fish-eye lenses for 35 mm and
medium format cameras are of the equidistant projection type, so
that there is no contraction of the image near the edges.  The
NORSTAR cameras from KEO Consultants use this optical system.
However, a 1999/08/31 fax from Bob Eather contains a plot that is
somewhere between equisolid and orthographic.  We must expect that
in general the mapping function will be non-ideal, and may vary
from camera to camera.  A test of low-order polynomial
approximations to the standard projections was found to be
unsatisfactory. Further tests showed that low order order rational
polynomials were much superior.  The simple form
 \begin{equation}
   \gamma^\prime = {{a_1 \gamma}\over{1+b_2 \gamma^2 }}
  \label{eq:gammafunc}
 \end{equation}
gave fits good to approximately $0.1^\circ$ while a slightly more
complicated form
 \begin{equation}
   \gamma^\prime = {{a_1 \gamma + a_3 \gamma^3}\over{1+b_2 \gamma^2 + b_4 \gamma^4}}
   \label{eq:gammafunc2}
 \end{equation}
gave maximum errors on the order of $0.01^\circ$.  The right panel
of Figure \ref{fig:fisheyefunctions} contains a plot of the errors
as a function of angle. These results suggest that equations of
the form \ref{eq:gammafunc} or \ref{eq:gammafunc2} should be
adequate for approximating the actual camera mapping function.

\begin{figure}
 \centering
  \includegraphics[width=0.4\columnwidth]{../fig/fisheye_functions}
  \includegraphics[width=0.4\columnwidth]{../fig/fisheye_ratpolyfits}
    \caption[Fish-eye functions]
   {Left Panel: Plots of five ideal ``fish-eye functions'' mapping angle from optical
     axis to radial displacement.  Right Panel: Errors in
     approximating functions with equation \ref{eq:gammafunc2}.
     \label{fig:fisheyefunctions}  }
\end{figure}


%Note that we're assuming the CCD is perfectly flat in the image
%plane.  This should be very nearly true, and I don't want to think
%about how to deal with it if it's not.


%%%%%%%%%%%%%%%%%%%%%%%%%%%%%%%%%%%%%%%%%%%%%%%%%%%%%%%%%5
%
\subsection{CCD Pixels}

The detector is assumed to be a rectangular array of size $L_x
\times L_y$ with $N_x \times N_y$ rectangular pixels.  The pixels
will be indexed with integers ranging from $x_{CCD} = 0\ldots
N_x-1$ , $y_{CCD} = 0\ldots N_y-1$. For the purposes of this
document we will assume that the pixels completely
fill\footnote{
 %
In practice there may be dead space between the light sensitive
regions.  This would be important for calculating response to
luminous flux.  However, that does not matter for the geometrical
issues considered in this document, so long as the pixels are
evenly spaced. }
 %
the CCD and thus have dimensions $d_x={L_x \over N_x}$, $d_y= {L_y
\over N_y}$.  Finally, it will also be convenient to define the
pixel aspect ratio $a_{y/x} = {d_y / d_x}$.

We will typically be interested in the location with respect to
the CCD position corresponding to the ``optical axis''
$(x_o,y_o)$. Furthermore, it is more convenient to work with a
dimensionless quantity produced by normalizing with pixel size.
The resulting ``pixel radius''
 \begin{equation}
  r_p = d / d_x = \sqrt{ (x - x_o)^2 + a_{y/x}^2 \, (y - y_o)^2 }
  \label{eqn:pixelradius}
 \end{equation}
and ``pixel angle''
 \begin{equation}
  \eta_p = \tan \left( a_{y/x} {{y-y_o}\over{x-x_o}} \right)
  \label{eqn:pixelangle}
 \end{equation}
are useful coordinates for relating CCD response to the optical
system.  Inverting these to recover pixel indices is
straightforward
 \begin{eqnarray}
  x &=& x_o + r_p \; \cos\eta_p \\
  y &=& y_o + r_p \; \sin\eta_p
 \end{eqnarray}


 \begin{figure}[htb!]
 \centering
 \setlength{\unitlength}{2.0cm}
 \begin{picture}(5,4)

 \color[gray]{0.5}
 \multiput(1,1)(0.0,0.1){31}{\line(1,0){3.75}}
 \multiput(1,1)(0.125,0.0){31}{\line(0,1){3}}
 \color{black}

 \thicklines

 \put(1.5,0.7){\vector(1,0){2}}
 \put(2,0.3){$\mathbf{X}_{CCD}$}

 \put(0.7,1.5){\vector(0,1){2}}
 \put(0.1,1.2){$\mathbf{Y}_{CCD}$}

 \put(2.45,2.45){\circle*{0.1}}
 \put(1.8,2.1){$x_o,y_o$}

 \put(2.47,2.48){\vector(1,1){1.1}}
 \put(3.7,3.65){\circle*{0.1}}
 \put(3.8,3.7){$x,y$}
 \put(2.9,3.2){$r_p$}

% \put(2.5,2.0){\oval(1.5,1.5)[tr]}
 \qbezier(3.05,3.05)(3.55,3.05)(3.55,2.45)
 \put(3.2,2.7){$\eta_p$}

 \qbezier[20](2.45,2.45)(3.0,2.45)(3.8,2.45)

\end{picture}
 \caption{An illustration of the coordinate
systems and conventions used for a CCD.  The optical axis
$(x_0,y_0)$ is deliberately placed off-center.  In practice it
will be near, but not necessarily exactly at, the CCD center
point.}
\end{figure}

\newpage
%%%%%%%%%%%%%%%%%%%%%%%%%%%%%%%%%%%%%%%%%%%%%%%%%%%%%%%%%%%%%%%%%%%%%%
\section{Data}

All-sky imagers typically use narrow band filters to isolate
wavelengths of auroral interest while minimizing background
contamination. The resulting images will contain light from
celestial sources, but this may be obscured by auroral and airglow
contributions.  In addition to the auroral filters, many systems
also take data using one or more background channels with
passbands in spectral regions where no auroral luminosity is
expected.  These background channels are essential for accurate
photometric work with faint emissions, but also produce useful
``star frames''. NORSTAR ASIs typically have a background 2.0 nm
filter centered at 480.0 nm (sometimes denoted as {\code BACKGR}).
Exposures of a second or more at this wavelength will image
several hundred stars if the viewing conditions are good.
High-pass (near-infrared) and open filters may also be useful,
although no data from these will be used for this report.

\begin{figure}[hbt!]
 \centering
  \includegraphics[width=0.95\columnwidth]{../fig/onenight}
    \caption[A one hour sequence.]
   {A one hour sequence with 59 background frames from Gillam
   during 06-07 UT December 21, 2001.  Edge-enhancement has
   been used to emphasize star tracks.
   The Big Dipper is visible to the left of Polaris, which is
   a single bright point source at pixel coordinates 122,67.
   An overhead cable can be seen in the upper portion
   of the frame, with a GPS antenna and light-shield in the lower left corner.
     \label{fig:onenight}  }
\end{figure}


\newpage
%%%%%%%%%%%%%%%%%%%%%%%%%%%%%%%%%%%%%%%%%%%%%%%%%%%%%%%%%%%%%%%%%%%%%%%%%%
\section{Fitting} \label{sec:fitting}

Three parameters are required to fully specify the instrument
orientation, two more for the CCD alignment, and at least one more
for the camera optics.  Determining these six (or more) parameters
is the central focus of this report.  Previous sections contain
all the information required to convert from pixel coordinates to
a celestial reference frame.  The problem is now to select a set
of orientation and instrument parameters that is in some sense
optimal.  A practical solution requires the definition of some
quality function which provides a quantifiable way of ranking
results.  Using this quality function, standard algorithms may be
used to search for the ``best'' set of parameters.  In practice it
is relatively simple to find a good solution, but much harder to
be sure that it is the best, as parameter space may contain many
local minima, and convergence to a global minimum may be
impossible to guarantee.


 \subsection{Initial Values}

Parameter estimates can be determined from celestial sources with
known locations. Initially, only sources at relatively small
zenith angles ($<45^\circ$) should be used.  Working with low
elevation targets may provide better constraints on camera
orientation, but requires more complex models of the optical
function.  Unless there is a reason to do otherwise, it is useful
to assume that the camera is roughly level, so $tilt=0, pitch=0$.
In many cases the system will also be roughly north-south aligned,
in which case $yaw \approx 0$.

 \begin{table}[htb!]
   \begin{center}
   \hrule
   \footnotesize
   \verbatiminput{c:/data/norstar/gill/gill_19700101_polaris.asi_info}
   \normalsize
   \hrule \vspace*{-3ex}
   \end{center}
     \caption[Initial values]
   {Camera parameters for Gillam estimated from a single star frame at 11:12 UT
   on December 23, 2001.
     \label{tab:initial_values}  }
   \hrule
 \end{table}

Viewed from Gillam,  Polaris is nearly stationary at azimuth
$0.0^\circ$ and a zenith angle of $33.6^\circ$.  Examination of a
frame sequence from 2001/12/23 shows a peak brightness at
approximately\footnote{More precision could be obtained by fitting
or centroiding, but that is not warranted at this stage.} $x=122,
y=67$.  Eta Ursa Major is convenient as a second source, as it is
easily identified (the last bright point in the handle of the Big
Dipper) and travels over a useful range of sky.  At 11:21 UT Eta
Ursa Major was at pixel $x=86, y=117$ which should correspond to
azimuth $90.1^\circ$ and zenith angle $24.3^\circ$. Assuming
approximate north-south alignment, the CCD position of the optical
axis is $x_0\approx 122, y_0\approx 117$ and the first optical
parameter is
 \begin{equation}
  a_1 \approx {{122-86}\over 24} \approx 1.5 \qquad\qquad
  a_1\approx {{117-67}\over 34} \approx 1.47
 \end{equation}
so an initial value of 1.48 is not unreasonable.  Using these
values (summarized in Table \ref{tab:initial_values}) produces
results shown in Figure \ref{fig:bigdipper_smeared}, which are
obviously smeared and offset from the correct locations.  Using
improved values (Table \ref{tab:monte_values}) gives the results
in Figure \ref{fig:bigdipper} where stars are better localized and
have offsets of less than a degree.  However, other regions of the
sky (not shown) still have small mapping errors.

\begin{figure}[htb!]
  \includegraphics[width=0.95\columnwidth]{../fig/bigdipper_smeared_image.eps}
    \caption[Poor quality composite image of the Big Dipper]
   {Composite image of the Big Dipper using parameters from Table
   \ref{tab:initial_values}.
     \label{fig:bigdipper_smeared}  }
\end{figure}

 \subsection{Quality of Fit}

Qualitative assessment of the results produced by any set of
parameters is simple.  Images projected into celestial coordinates
are obviously ``better'' if stars are properly focused (Figure
\ref{fig:bigdipper}) and worse if stars are smeared or offset from
their correct positions (Figure \ref{fig:bigdipper_smeared}).
Visual examination is sufficient for identifying a good fit and
even for determining whether a change in parameters provides
further improvement.  Quantifying this is more difficult.  Several
methods were explored for this report, with the goal being to
determine a robust and easily calculated quantity that accurately
reflects the quality of a projection.

The current algorithm is encapsulated in the IDL program
{\verb!asi_map_focus.pro!} included in \S \ref{sec:idl_code}. It
seems to work reasonably well, but could probably be improved.


 \subsection{Random Searching}

As noted previously, there may not be a single global minimum.  A
general exploration of parameter space can help determine where
local minima may be located, and validate the choice of initial
parameters.  Systematic searches of a $d$ dimensional parameter
space require $O(n^d)$ evaluations of the quality function, where
$n$ is the number of divisions per dimension.  A comprehensive
high-resolution multi-dimensional search may require unreasonable
amounts of computing power.  An alternative is to evaluate the
quality function at random points in parameter space.  Such a
Monte Carlo search was carried out with 25,000 points over a wide
range of parameters, with a subsequent 25,000 points over a
smaller parameter range near the optimal fit.  Results are shown
in Figure \ref{fig:monte_scatter}.  All parameters exhibit a clear
single minimum.

\begin{figure}[htb!]
  \includegraphics[width=0.95\columnwidth]{../fig/monte_scatter.eps}
    \caption[Monte Carlo parameter search]
   {Results from a Monte Carlo search of 6-dimensional parameter space.
    Darker symbols are from the first stage of a wide search (25,000 points),
    lighter symbols from the second stage (25,000 points) concentrating on the
    region of best fit.  Quality of fit is indicated on the
    y-axis, with small (more negative) numbers corresponding to
    better fits.
     \label{fig:monte_scatter} }
\end{figure}

While not as time consuming as an exhaustive search, this monte
carlo survey required nearly 24 hours on a desktop computer.  This
is not an issue if parameter estimates only need to be
re-determined for a few cameras each year.  Increasing
computational resources are likely to keep pace with expansion of
the NORSTAR project; some improvements in code efficiency are also
possible.

 \begin{table}[htb!]
   \begin{center}
   \hrule
   \footnotesize
   \verbatiminput{c:/data/norstar/gill/gill_20010801_polaris.asi_info}
   \normalsize
   \hrule \vspace*{-3ex}
   \end{center}
     \caption[Initial values]
   {Camera parameters for Gillam from a Monte Carlo search.
     \label{tab:monte_values}  }
   \hrule
 \end{table}


\begin{figure}[htb!]
  \includegraphics[width=0.95\columnwidth]{../fig/bigdipper_image.eps}
    \caption[Composite image of Ursa Major]
   {Projection of 54 background frames from 06-12UT 2001/12/21
   onto a common GEI grid for the region containing Ursa Major.
   Pointing accuracy is on the order of $1/2^\circ$.
     \label{fig:bigdipper}  }
\end{figure}


 \subsection{Systematic Optimization}
There are many algorithms designed to search for minima in a
multi-dimensional parameter space.  The {\sf IDL} routine {\tt
amoeba} \citep[see \S 10.4]{numerical_recipes} was used for this
report.  This algorithm does not require derivatives, which are
not directly available here and would have to be estimated
numerically.  It is not particularly efficient, but has proven to
be extremely robust.  Other, ``better'', routines tended to
proceed inappropriately when confronted with bad data.  Such
behavior is unacceptable for a final goal of automatic fitting
without human intervention.


\begin{figure}[htb!]
  \includegraphics[width=0.95\columnwidth]{../fig/fit_1d.eps}
  \caption{Quality of fit as a function of the optical parameter $a_1$.
   There is an optimal minimum near $a_1=1.44$, but fitting algorithms
   may be confused by the local minimum at $a_1=1.49$.
     \label{fig:fit_1d}  }
\end{figure}


\begin{figure}[htb!]
  \includegraphics[width=0.95\columnwidth]{../fig/fit_2d.eps}
    \caption{Contour plot of fit quality for various estimates of
    the CCD optical axis $(x_0,y_0)$.  A clear minimum is evident
    near $(123,119)$.
     \label{fig:fit_2d}  }
\end{figure}


\clearpage
\section{Viewing Conditions}

If camera orientation and optical parameters are known, celestial
sources may be used for other purposes.  One particularly useful
application is determination of viewing conditions, as clouds and
haze will reduce or eliminate the photon flux from stars. A
measure of star brightness relative to a surrounding star-free
region consequently provides an indication of atmospheric
transmission. This indicator can be determined automatically and
may be sufficiently quantitative as to distinguish between clear
sky, haze, and thin cloud.  This is in contrast to visual
examination of star frames, which is fundamentally qualitative and
labor intensive.

\begin{figure}[hbt!]
  \includegraphics[width=0.95\columnwidth]{../fig/clearsky_polaris}
    \caption[Clear sky estimates using Polaris]
   {Background brightness (squares) and excess counts due to
   Polaris (stars).
     \label{fig:clearsky_polaris}  }
\end{figure}

Polaris is often used as a target for photometric determination of
viewing conditions because it is essentially fixed in the sky and
is isolated from other bright stars.  These properties are also
useful for ASI applications.  Even without accurate knowledge of
the camera optics and orientation, it is straightforward to
identify the region of pixels containing Polaris, as well as an
adjacent star-free region.  Visual examination of a sequence of
frames during several hours of 2001/12/23 suggests hazy conditions
during the first few hours of this interval after which the sky
appears to be clear. The sequence of events is consistent with the
time series shown in Figure \ref{fig:clearsky_polaris}.

The primary drawback of using Polaris is that it provides an
indication of viewing conditions at only a single point.  By using
more stars it would be possible to obtain a full sky measure of
visibility.  This of course requires a sufficiently accurate
knowledge of camera orientation and optics to track stars over the
course of a night.  It is also necessary to use only stars that
are visible over the entire evening, or somehow account for their
rising and setting.

\begin{figure}[htb!]
  \includegraphics[width=0.95\columnwidth]{../fig/clearsky_bigdipper}
    \caption[Clear sky estimates using Ursa Major]
   {Background brightness and excess counts due to
   Ursa Major.
     \label{fig:clearsky_bigdipper}  }
\end{figure}

A test of feasibility was carried out with the ``Big Dipper''
(Ursa Major). Frames from the night of 2001/12/23 were projected
onto a GEI grid with the average shown in figure
\ref{fig:bigdipper}. Star brightness was estimated by adding all
values within $2^\circ$ of the nominal star position. Background
was determined from regions outside these boundaries. Results are
shown in Figure \ref{fig:clearsky_bigdipper}, and are clearly
similar to those in figure \ref{fig:clearsky_polaris}. However,
using the Big Dipper provides an indication of viewing conditions
over a much larger area ($16^\circ \times 16^\circ$) with enhanced
signal-to-noise due to the larger number of sources. Increased
spatial coverage could be obtained by increasing the number of
target stars.

%%%%%%%%%%%%%%%%%%%%%%%%%%%%%%%%%%%%%%%%%%%%%%%%%%
\section{Summary}
%
This report contains all the information required to project ASI
frames into celestial coordinates.  When combined with a measure
of fit quality, this allows orientation and other instrument
characteristics to be determined with a minimum of human
intervention.  Estimates of viewing conditions can also be
obtained automatically.

\begin{figure}[htb!]
  \includegraphics[width=0.95\columnwidth]{../fig/composite_skymap.jpg.eps}
   \caption{Composite image of the northern celestial sphere using parameters from Table
   \ref{tab:initial_values}.
     \label{fig:composite_skymap}  }
\end{figure}

 \subsection{Suggested Implementation}
Preliminary estimates of instrument parameters should be obtained
from visual examination of single star frames in combination with
{\it a priori} knowledge.  Results should be confirmed with an
exhaustive monte-carlo search throughout a wide region of
parameter space.  Next, systematic searching would be used to
determine optimal parameters.  This should be repeated using
several different initial conditions in order to avoid local
minima.  Finally, large scale composite star maps should be
created to verify the global quality of fit.  Figure
\ref{fig:composite_skymap} contains an example of such a global
fit based on the parameters in Table \ref{tab:monte_values}.
Quality of focus is fairly good in most regions of the sky, but
stars near the horizon are slightly smeared.  Some improvement is
clearly possible.

% \subsection{Future Improvements}


\newpage
%%%%%%%%%%%%%%%%%%%%%%%%%%%%%%%%%%%%%%%%%%%%%%%%%%%%
\section{Additional Information}

 \subsection{Jupiter}

 \tiny
 \verbatiminput{gillam_jupiter_ephemeris.txt}
 \normalsize

 \subsection{Bright Stars}

 \begin{table}[htb!]
   \begin{center}
   \fontfamily{cmsf} \fontsize{8}{8} \selectfont
   \begin{tabular}{ccrrr} \hline
Name & Catalog &   Right  &  Declination &  Visual \\
 & Number &  Ascension  &   &  Magnitude \\ \hline
Alp UMi &     424 &   37.953 &   89.264 &   2.02 \\ 
Bet UMi &    5563 &  222.676 &   74.156 &   2.08 \\ 
Alp Cep &    8162 &  319.645 &   62.586 &   2.44 \\ 
Alp UMa &    4301 &  165.932 &   61.751 &   1.79 \\ 
Gam Cas &     264 &   14.177 &   60.717 &   2.47 \\ 
Bet Cas &      21 &    2.295 &   59.150 &   2.27 \\ 
Alp Cas &     168 &   10.127 &   56.537 &   2.23 \\ 
Bet UMa &    4295 &  165.460 &   56.382 &   2.37 \\ 
Eps UMa &    4905 &  193.507 &   55.960 &   1.77 \\ 
Zet UMa &    5054 &  200.981 &   54.925 &   2.27 \\ 
Gam UMa &    4554 &  178.457 &   53.695 &   2.44 \\ 
Gam Dra &    6705 &  269.152 &   51.489 &   2.23 \\ 
Alp Per &    1017 &   51.081 &   49.861 &   1.79 \\ 
Eta UMa &    5191 &  206.885 &   49.313 &   1.86 \\ 
\rowcolor[gray]{0.8}  Alp Aur &    1708 &   79.173 &   45.998 &   0.08 \\ 
Alp Cyg &    7924 &  310.358 &   45.280 &   1.25 \\ 
Bet Aur &    2088 &   89.882 &   44.947 &   1.90 \\ 
Gam1And &     603 &   30.975 &   42.330 &   2.26 \\ 
Bet Per &     936 &   47.042 &   40.956 &   2.12 \\ 
Gam Cyg &    7796 &  305.557 &   40.257 &   2.20 \\ 
\rowcolor[gray]{0.8}  Alp Lyr &    7001 &  279.235 &   38.784 &   0.03 \\ 
Bet And &     337 &   17.433 &   35.621 &   2.06 \\ 
Eps Cyg &    7949 &  311.553 &   33.970 &   2.46 \\ 
Alp Gem &    2891 &  113.650 &   31.888 &   1.98 \\ 
Alp And &      15 &    2.097 &   29.091 &   2.06 \\ 
Bet Tau &    1791 &   81.573 &   28.608 &   1.65 \\ 
Bet Peg &    8775 &  345.944 &   28.083 &   2.42 \\ 
Bet Gem &    2990 &  116.329 &   28.026 &   1.14 \\ 
Alp CrB &    5793 &  233.672 &   26.715 &   2.23 \\ 
        &    5958 &  239.876 &   25.920 &   2.00 \\ 
Alp Ari &     617 &   31.793 &   23.462 &   2.00 \\ 
\rowcolor[gray]{0.8}  Alp Boo &    5340 &  213.915 &   19.183 &  -0.04 \\ 
\rowcolor[gray]{0.8}  Alp Tau &    1457 &   68.980 &   16.509 &   0.85 \\ 
Gam Gem &    2421 &   99.428 &   16.399 &   1.93 \\ 
Alp Peg &    8781 &  346.190 &   15.205 &   2.49 \\ 
Bet Leo &    4534 &  177.265 &   14.572 &   2.14 \\ 
Alp Oph &    6556 &  263.734 &   12.560 &   2.08 \\ 
Alp Leo &    3982 &  152.093 &   11.967 &   1.35 \\ 
Eps Peg &    8308 &  326.047 &    9.875 &   2.39 \\ 
\rowcolor[gray]{0.8}  Alp Aql &    7557 &  297.696 &    8.868 &   0.77 \\ 
\rowcolor[gray]{0.8}  Alp Ori &    2061 &   88.793 &    7.407 &   0.50 \\ 
Gam Ori &    1790 &   81.283 &    6.350 &   1.64 \\ 
\rowcolor[gray]{0.8}  Alp CMi &    2943 &  114.825 &    5.225 &   0.38 \\ 
Del Ori &    1852 &   83.002 &    0.299 &   2.23 \\ 
Eps Ori &    1903 &   84.053 &   -1.202 &   1.70 \\ 
Zet Ori &    1948 &   85.190 &   -1.943 &   2.05 \\ 
\rowcolor[gray]{0.8}  Bet Ori &    1713 &   78.635 &   -8.202 &   0.12 \\ 
Alp Hya &    3748 &  141.897 &   -8.659 &   1.98 \\ 
Kap Ori &    2004 &   86.939 &   -9.670 &   2.06 \\ 
\rowcolor[gray]{0.8}  Alp Vir &    5056 &  201.298 &  -11.161 &   0.98 \\ 
Eta Oph &    6378 &  257.595 &  -15.725 &   2.43 \\ 
\rowcolor[gray]{0.8}  Alp CMa &    2491 &  101.287 &  -16.716 &  -1.46 \\ 
Bet CMa &    2294 &   95.675 &  -17.956 &   1.98 \\ 
Bet Cet &     188 &   10.898 &  -17.987 &   2.04 \\ 
\hline
\end{tabular}

   \normalsize
   \end{center}
 \caption{Sources from the Yale Bright Star Catalog with visual magnitudes
          less than 2.5, sorted by declination.  Particularly bright sources
          are marked with gray shading.  Sources with declinations less than
          $-20^\circ$ are not included.}
 \end{table}
\newpage


  \begin{landscape}
 \subsection{IDL Code} \label{sec:idl_code}
  \fontfamily{cmtt} \fontsize{6}{5} \selectfont
  \setlength{\columnseprule}{0.4pt}
  \begin{multicols}{2}
  {\large \bf asi\_map\_init.pro}
   \verbatiminput{../idl/asi_map_init.pro}
  \end{multicols}
  \normalsize
  \end{landscape}

%%%%%%%%%%%%%%%%%%%%%%%%%%%%%%%%%%%%%%%%%%%%%%%%%%%%%%%%%%%%%%%%%%%%%%
%\nocite{*}

\newpage
\bibliography{asi_map}

\end{document}
