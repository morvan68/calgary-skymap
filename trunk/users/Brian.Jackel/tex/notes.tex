\documentclass[12pt,twoside,openright]{report}
%\include{datalogy_preamble}
\usepackage{amssymb,amsmath}
\usepackage{mathrsfs}  % provides $\mathscr E$ for EMF symbol
\usepackage[]{graphicx}
\DeclareGraphicsExtensions{.eps,.pdf,.png,.jpg,.r.eps,.r.pdf,.gif.png}
%\graphicspath{{../fig/}}
\usepackage{color}
\usepackage{url}
%\usepackage{subeqn}

% Make a larger page (shrink the margins)
%
\setlength{\headheight}{14pt}  % default 12pt is too small for fancyhdr
\setlength{\textwidth}{6.7in}
\setlength{\textheight}{9.0in}
\setlength{\evensidemargin}{0.0in}
\setlength{\oddsidemargin}{0.0in}
\topmargin -0.5in
\footskip 0.5in

\begin{document}

\subsection{Rotation about an arbitrary axis}

\url{http://tom.cs.byu.edu/~455/hw/rotation.pdf}


A vector $\vec v = (x,y,z)$ is rotated by an angle $\theta$ about an axis given by the unit vector $\hat n = (a,b,c)$.  It can be shown that the $3\times3$ rotation matrix has the form
\begin{equation}\label{eqn:general_rotation_matrix}
\begin{bmatrix}
  a^2 + (1-a^2)\cos\theta & a b (1-\cos\theta) - c \sin\theta & a c (1-\cos\theta) + b \sin\theta \\
  a b (1-\cos\theta) +  \sin\theta & b^2 + (1-b^2)\cos\theta & b c (1-\cos\theta) - a \sin\theta \\
  a c (1-\cos\theta) - b \sin\theta & b c (1-\cos\theta) + a \sin\theta & c^2 + (1-c^2)\cos\theta \\
\end{bmatrix}
\end{equation}

\subsubsection{Alternative}
A more geometrically intuitive\footnote{but computationally inferior?} approach starts with noting that the vector $\vec v$ can be decomposed into parts parallel and perpendicular to the rotation axis $\hat n$
\begin{equation}\label{}
\vec v = {\vec v}_\parallel + {\vec v}_\perp
\end{equation}
where
\begin{subequations}\label{}
\begin{eqnarray}
{\vec v}_\parallel &=& \hat n (\hat n \cdot \vec v) \\
{\vec v}_\perp &=& \vec v - {\vec v}_\parallel
\end{eqnarray}
\end{subequations}
The process of rotation has no effect on the parallel component, with all changes in the plane perpendicular to the rotation axis.  The rotated perpendicular component $= {\vec v}_\perp^\prime$ will be in a plane is spanned by ${\vec v}_\perp$ and a second vector defined as
\begin{equation}\label{}
\vec u = \hat n \times {\vec v}_\perp = \hat n \times {\vec v}   \qquad\qquad \mathbf{because}\; \hat n \times {\vec v}_\parallel = 0
\end{equation}
with the the standard right-handed coordinate system so that ${\hat v}_\perp \times \hat u = \hat n$ and a positive angle $\theta$ rotates ${\vec v}_\perp$ towards $\vec u$.  The rotated perpendicular component  can be written in terms of two orthogonal basis vectors and the rotation angle where  $|{\vec u}| = |{\vec v}_\perp| = |{\vec v}_\perp^\prime|$, so
\begin{equation}\label{}
{\vec v}_\perp^\prime = {\vec v}_\perp \cos\theta + \vec u \sin\theta
\end{equation}

\noindent The final rotated vector can be written in several different ways
\begin{subequations}\label{}
\begin{eqnarray}
{\vec v}^\prime &=& {\vec v}_\parallel + {\vec v}_\perp^\prime \\
       &=& {\vec v}_\parallel + {\vec v}_\perp \cos\theta + \vec u \sin\theta \\
       &=& {\vec v} + {\vec v}_\perp (\cos\theta-1) + \vec u \sin\theta
\end{eqnarray}
\end{subequations}

I suspect that the computational requirements are sub-optimal...
\begin{enumerate}
  \item Calculate ${\vec v}_\parallel$: dot product (3 mult, 2 add) followed by vector-scalar multiplication (3 mult) = 6 multiply, 2 addition
  \item Calculate ${\vec v}_\perp$: vector subtraction = 3 addition
  \item Calculate ${\vec u}$: cross product = 6 multiply, 3 addition
  \item Combine terms = 6 multiply + 6 addition
\end{enumerate}
...certainly less elegant than a matrix multiplication

\section{IDL code}

Would it be worth introducing a ``skymap\_aim'' object that is an overloaded vector with za/az options?
Or just overload ``skymap\_vector'' spherical conversions?

There are 4 combinations:
 1) theta/phi        co-latitude/longitude , right ascension/declination
 2) \bar{theta}/phi  latitude/longitude
 3) theta/\bar{phi}  zenith angle/azimuth (clockwise from north)
 4) \bar{theta}/\bar{phi} elevation/azimith
 5) some wacky camera convention?

Try keyword SPHERICAL=[1|2|4|6] where bit 0 is basic conversion, bit 1 is co-latitude, bit 2 is co-azimuth.

For Cartesian $x,y,z$ the azimuth is the angle in the x-y plane relative to the x-axis and positive towards +y
$$ \phi = \atan(y,x) $$
but for geographic East,North,Up azimuth is clockwise from north so we get
$$ \phi = \atan(x,y) $$
with the alternative being North,East Down
$$ \phi = \atan(y,x) $$
or left-handed coordinates


\end{document} 