\documentclass[11pt]{article}

\usepackage{graphicx, epsfig}
\usepackage{amsmath, amssymb, latexsym}
\usepackage[english]{babel}
\usepackage{amssymb}   
%\usepackage{graphicx}
%\usepackage{float} 






% Make a larger page (shrink the margins)
%
\setlength{\textwidth}{6.7in}
\setlength{\textheight}{9.0in}
\setlength{\evensidemargin}{0.0in}
\setlength{\oddsidemargin}{0.0in}
\topmargin -0.5in
\footskip 0.5in



\title{Progress Report}
\author{Eric}
\begin{document}
\maketitle
\medskip
\section{Licensing}
\hspace{0.5cm}Looking around a bit to compare licensing platforms I think you are using the best one. According to the website http://www.dwheeler.com/essays/gpl-compatible.html GPLs account for 50-80\% of projects hosted on project hosting websites. I did find a couple pages discussing how the open source project percentage share of GPLs has been declining. However, in real terms the amount of projects using GPL is still increasing. Even searching around a website like www.freecode.com there are more projects tagged under the GPL than any other open source license.  The only other open source license I think you would want to consider would be the Apache 2.0, which accounts for about 20-30\%  of open source licenses. However, using a GPL will still allow people to use and distribute any code you publish, they just must redistribute any merged licensed code under the GPL. If the goal is to make your code easily accessible to the largest group possible, sticking with GPL is the way to go.

Couple websites:\\
http://www.itworld.com/it-managementstrategy/233753/gpl-copyleft-use-declining-faster-ever \\
http://blogs.the451group.com/opensource/2011/12/15/on-the-continuing-decline-of-the-gpl/

\section{IDL}

Did some programming in IDL, which are uploaded onto subversion. Made a script which returns the rotated vector and rotation matrix after inputing the rotation axis as well as the angle theta. The alternative method code looks simpler, but computationally they seem the same i.e. both take 0 seconds to run. I also played around with producing a code that returns the rotation axis and theta when given two vectors. I played around with a bunch of methods to try and get the best results. One script returns the rotation matrix and axis for the minimal possible rotation (smallest theta). I don`t know if there is much use for it but I just wanted to gain experience using IDL. So really even all my half working scripts weren't an entire waste of time. Definitely think I am a lot more comfortable working in IDL now. Anyways, going to put some working scripts up on Subversion.


\section{Stellar Calibration}

Just read over everything listed there. Found a SkyCalc gui manual at http://mdm.kpno.noao.edu/Manuals/doc/guimanual.html which was quite helpful in getting SkyCalc to function. Felt it was kind of odd that I had never come across the distinct definitions for both radiometry and photometry with my astrophysics classes but I am now well versed. 




\end{document}
\end{article}