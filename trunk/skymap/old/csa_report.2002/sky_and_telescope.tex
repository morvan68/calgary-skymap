\documentclass[11pt,twoside]{article}   % pre-print version

\usepackage[active]{srcltx} % SRC Specials: DVI [Inverse] Search
\usepackage[dvips]{graphicx} % for including figures
\usepackage[dvips]{color}
\usepackage[square]{natbib}  % AGU style references

\usepackage{verbatim}   %to include other text files and source code listings
\usepackage{fancyhdr}   %to play with top of page headers
\usepackage{lastpage}   %to allow "page 1 of N"
\usepackage{longtable}  %for tables that span multiple pages
\usepackage{sectsty}    %to modify section headings
%\usepackage{newcent}    %use this to get different looking output
\usepackage{times}    %use this to get different looking output
\usepackage{fancybox}
\usepackage{multicol}
\usepackage{colortbl}
\usepackage{lscape}  % to allow switching between portrait and landscape
\usepackage{amsfonts}
%\usepackage{sublabel}

% Make a larger page (shrink the margins)
%
 \setlength{\textwidth}{6.0in}
 \setlength{\textheight}{8.5in}
 \setlength{\oddsidemargin}{0.5in}
 \setlength{\evensidemargin}{0.5in}
 \setlength{\voffset}{-0.5in}
% \setlength{\topskip}{0.5in}
% \setlength{\footskip}{1.5in}

% The  sectsty  package lets us play games with the section headings
 %\allsectionsfont{\sffamily\noindent\shadowbox}
 %\sectionfont{\sffamily\noindent\ovalbox}  % got carried away here
% \allsectionsfont{\color[gray]{0.9}\sffamily\noindent\shadowbox}
  \allsectionsfont{\color[gray]{0.2}\sffamily\noindent\colorbox[gray]{0.9}}

 \pagestyle{fancy}
 \lfoot{\today}
 \cfoot{}
 \rfoot{\thepage\ of \pageref{LastPage}}
 \renewcommand{\headrulewidth}{0.4pt}
 \renewcommand{\footrulewidth}{0.4pt}

 \newcommand{\note}[1]{\marginpar{\small #1}}
 \setlength{\columnseprule}{.4pt}

% Use this to have section numbers in left margin.
%
 \makeatletter
 \def\@seccntformat#1{\protect\makebox[0pt][r]{\csname the#1\endcsname\quad}}
 \makeatother

\begin{document}

\begin{center}
\section*{\quad Auroral studies by starlight\quad }

% \footnotetext{An alternate title might be ``Auroral studies by starlight''}

 \parbox{4in}{\centering
     Brian Jackel, Trond Trondsen, and Eric Donovan \\[2ex]
     \it University of Calgary \\
     Institute for Space Research }
\end{center}

\begin{multicols}{2}
Stars and the aurora have almost nothing in common, except that
both can be seen in the clear night sky. While stars remain steady
and fixed in the celestial sphere, the aurora flicker and dance,
reflecting the dynamic interaction of the solar wind with the
Earth's magnetic field.

Aurora are often highly variable and quite bright relative to
every celestial object except the sun and moon.  Consequently,
professional astronomers see auroral light as just another source
of noise to be avoided or removed. In turn, auroral scientists
 %\footnote{Trond: I vaguely remember that you used stars to align PAI.
 %However, I can't see a good way to work that in.}
typically act as if stars don't even exist, as their contribution
to most measurements is negligible.  However, a careful analysis
of starlight provides information that can be used in the field of
auroral research.

At the University of Calgary Institute for Space Research we study
the aurora in a variety of ways.  Satellites in space, rocket
flights through the aurora, and observations from the ground each
tell us something important about the cause of the aurora and its
effect on the upper atmosphere.

%\end{multicols}
\begin{figure*}[htb!]
% \centering
  \includegraphics[width=1.95\columnwidth]{../fig/onenight}
    \caption[A one hour sequence.]
   {A one hour sequence with 59 background frames from Gillam,
   Manitoba during 06-07 UT December 21, 2001.  Edge-enhancement has
   been used to emphasize star tracks.
   The Big Dipper is visible to the left of Polaris, which is
   a single point source at pixel coordinates 122,67.
   An overhead power line can be seen in the upper portion
   of the frame, with a GPS antenna and light-shield in the lower left corner.
     \label{fig:onenight}  }
\end{figure*}
%\begin{multicols}{2}

Some of the earliest auroral observations were carried out using
ground-based all-sky cameras.  As the name suggests, these took
photographs of the aurora through a ``fish-eye'' lens in order to
obtain pictures of the entire night sky.  This broad field of view
is essential for studies of large scale auroral patterns.  Today
we still use similar instruments updated with modern optics,
filters, and detector technology.

By far the most important advance for modern instruments is the
use of cooled charge-coupled device (CCD) detectors instead of
photographic film.  These not only provide enhanced sensitivity,
but also allow images to be easily stored in digital format and
processed in ways that were previously very difficult.

Many of our cameras also have image intensifiers to allow
operations at very low light levels.  This is important because we
typically use narrow-band (2 nm) interference filters to isolate
specific auroral wavelengths (e.g. 486.1, 557.7, and 630.0 nm).
Each exposure thus contains a small fraction of the total auroral
spectrum, but an even smaller amount of light due to other
``background'' sources such as stars.

A complete all-sky imager (ASI) system consists of a CCD camera,
filter-wheel, image intensifier, and miscellaneous optics and
electronics.  Our systems are controlled by PCs running Linux
which acquire images according to a pre-set schedule. These allow
operations to be automated so that data can be gathered at
unattended field sites in the remote Arctic during the long winter
nights.

Modern ASIs are superb scientific instruments for auroral studies.
They are, however, not very good telescopes.  All-sky lenses
provide an unparalleled field of view, but have certain
fundamental limitations.  It is difficult to achieve perfect focus
everywhere in the image, and the best compromise settings cause
point sources (such as stars) to be blurred by tens of
arc-minutes. Even with ideal optics, mapping a $180^\circ$ field
of view onto a $512{\times}512$ pixel CCD results in a resolution
limit of approximately ${1 \over 3}^\circ$. Using megapixel CCDs
might permit arc-minute resolution, but arc-seconds are definitely
out of the question.

Any particular star in the sky is thus imaged very poorly in each
recorded image.  This is partially compensated for by the fact
that \emph{every} celestial object above the horizon appears in
\emph{each} image.  It is (at least in theory) possible to project
each image into celestial coordinates by correcting for the
effects of the Earth's rotation. A sequence of images from such a
digitally despun telescope can be combined to produce a composite
image of the entire night sky. Data can be merged from an hour, a
night, or even an entire winter of observations to obtain very
high contrast images.

\begin{figure*}[htb!]
  \includegraphics[width=1.95\columnwidth]{../fig/bigdipper_smeared_image.eps}
    \caption[Poor quality composite image of the Big Dipper]
   {A composite image of the Big Dipper produced from 150 1-second exposures
   over a 7-hour interval, but using slightly erroneous pointing
   information. Circles of $1^\circ$ radius are centered where stars
   should be located.
     \label{fig:bigdipper_smeared}  }
\end{figure*}

\begin{figure*}[htb!]
  \includegraphics[width=1.95\columnwidth]{../fig/bigdipper_image.eps}
    \caption[Poor quality composite image of the Big Dipper]
   {As figure \ref{fig:bigdipper_smeared}, but using correct
   pointing information.
     \label{fig:bigdipper}  }
\end{figure*}


Projecting raw images into celestial coordinates requires accurate
knowledge of camera optics and orientation.  This brings us to a
crucial question: where are we looking? Ridiculous as this may
sound to even a novice stargazer, it is a real problem for us. Our
instruments are often installed at remote arctic sites under less
than ideal conditions.  When standing on the top of an unsteady
ladder it is difficult to ensure that a mounting frame is
perfectly level, much less aligned to the proper azimuth.
Fortunately, imperfect alignment is not really a problem, just so
long as we can find out what the orientation errors actually are.
It turns out that using stars as reference sources allows us to
determine instrument orientation very precisely.

For example, a poor estimate of instrument orientation might map
CCD pixels onto a right-ascension/declination grid as shown in the
first panel of figure \ref{fig:bigdipper_smeared}. Stars in the
big dipper (Ursa Major) are clearly offset from their proper
locations and smeared due to erroneous compensation for the
Earth's rotation. Results from a better estimate of camera
orientation are shown in figure \ref{fig:bigdipper}. Pointing
accuracy in this case is clearly better than ${1\over 2}^\circ$.
This would be considered poor for a typical narrow field
telescope, but is quite good for an all-sky imager with a
$180^\circ$ field of view, and more than sufficient for auroral
research.

The Big Dipper is an extremely useful feature for this kind of
work because it contains several bright stars which are always
visible in the Arctic night sky.  This constellation was known as
Tukturjuit (caribou) to the Inuit people, who used it to determine
the passage of time.  It is also interesting that Polaris is
virtually useless for navigation in the Arctic, as it is close to
the zenith and does not clearly indicate the direction of North.

\begin{figure*}[htb!]
  \includegraphics[width=1.90\columnwidth]{../fig/composite_skymap.jpg.eps}
  \caption{Composite of the northern celestial
  hemisphere constructed from nearly 1000 different all-sky images
  obtained during a 12-hour interval.
     \label{fig:composite_skymap}  }
\end{figure*}

Using the stars as celestial reference sources essentially solves
the problem of unknown instrument alignment. A computer program
can search through all possible orientations until it finds
parameters which focus stars to their correct locations in
celestial coordinates. Modern desktop computers can determine
optimal results overnight. This is sufficiently fast considering
that we only need to determine camera orientation once at each
site when the instrument mounting frame is first installed.

Correct camera orientation can be confirmed in some visually
appealing ways.  For example, figure \ref{fig:composite_skymap}
contains an image of the entire northern celestial hemisphere
which is a composite of almost one thousand individual exposures
gathered over nearly 24 hours. Stars in most regions are well
``focused'', with exceptions due to limitations in our knowledge
of certain optical characteristics of the instrument.  A composite
image built up from several weeks of observations would show the
apparent motion of the planets (such as Jupiter, which was the
brightest feature in the sky at this time).

%{\sf I want to emphasize the concept of a ``synthetic'' telescope
%a bit, and could also talk about super resolution stuff.}

Our ability to project all-sky images into celestial coordinates
has uses beyond determining instrument orientation. Stars also
provide luminosity standards which can be used for absolute
calibration of instrument sensitivity.  Although we carry out
extensive darkroom calibration, an independent confirmation using
stellar sources is extremely valuable, as it allows us to monitor
the performance of an auroral imager over years of operation in
the field.

Relative brightness also provides important information for
estimating viewing conditions.  Our current image catalog contains
more than a million images, and visual examination of each frame
for clouds is not feasible. Certain empirical algorithms for
automated image classification do exist, and are generally
effective, but a more physically meaningful approach would be
preferable.  This can be accomplished by first projecting images
into celestial coordinates, and then using standard bright star
catalogs to separate ``signal'' and ``background'' contributions.
Clouds will tend to remove photons from stellar sources, while
scattered light will appear more or less uniform across all
regions. Consequently, automated determination of cloud cover can
be obtained relatively easily by tracking the difference between
signal and background estimates.

%{\sf Figures need to be improved.  Pixel numbers on Figure 1 could
%be removed and contrast improved.  Figure 3 would look better if a
%full 24-hours were used.  Using two (or more) sites would increase
%the gee-whiz factor, but maybe not worth the time.}

Digital wizardry allows us to use auroral images in a variety of
ways that were never part of our original plans.  It is not,
however, clear whether any of this could play a useful role in
real astronomical observations.  Our need for image transformation
was a direct consequence of the extremely wide and fixed field of
view of our instrument.  A much easier way to acquire high quality
astronomical images would be to simply use a good quality narrow
field telescope with tracking capabilities. Long exposure times
could then be achieved without any computer assistance, and larger
mosaics built up over time. Still, perhaps our experience will
help inspire other new uses of CCD cameras in combination with
digital image processing.  We have certainly enjoyed our brief
excursion into the field of stellar cartography, continuing in the
long tradition of looking to the sky for immutable (and beautiful)
reference sources.

{\bf About the authors:} Drs. Jackel, Trondsen, and Donovan study
the aurora with optical and other techniques.  They are currently
operating a network of all-sky imagers in the Canadian Arctic,
more information can be found at {\small\tt
http://www.phys.ucalgary.ca/NORSTAR}.

\end{multicols}
\end{document}
